\section{Introduction}
This paper deals with the topic of running containerized applications on different types of hardware with different processor architectures.
Containers offer many advantages when it comes to running applications. They allow a high degree of independence from the base system and still offer a lower performance overhead than virtual machines. However, since they are also a form of virtualization, they are not independent of the processor architecture and must be adapted for each type of processor architecture. This is one of the problems that limits the cross-platform use of containers.

This is where this paper comes in. Each of the individual processor architectures comes with its advantages and disadvantages, which can also be tailored to specific application areas. This should be used for a smart way of load balancing to potentially reduce the energy consumption of the running applications. As a result, operations should be made more efficient and operating costs should be saved.
To answer this question, this paper will compare two processor architectures and determine their efficiency. To reduce complexity, only one type of application will be used for the test later in the paper. A large number of different applications with different hardware omissions significantly increases the complexity of the experiment and also makes it more difficult to determine the benefits of energy-efficient load balancing.


