\section{Methodology} \label{sec:methodology}

% TODO: Igor: Hier wäre vielleicht eine Einleitung gut, die erkärt, wie die Teile zusammenhängen

% Vorlage von Igors final paper:
% In this section we present the experimental setup for the power consumption measurements within a CPS. To test our concept, a CPS is built with the minimum requirements, consisting of a single sensor, actuator and controller (IoT framework). This means that one participant in this CPS measures the physical environment, another changes it, while the third is responsible for controlling the entire interaction between the sensor and actuator. This setup represents the smallest setup of interacting components that are needed to form a CPS, measure and change the physical world. After careful consideration, it was decided that this CPS was sufficient to show whether a difference is made by selecting a different design approach. It is also possible to extend the CPS with additional components to gain complexity, ensuring scalability. A measurement would be required for each ad- ditional component, but the measurement method would not change, regardless of the number of components within a CPS. In summary, we have chosen this simple CPS to proof the feasibility of the concept in the smallest possible CPS.


\subsection{Performance per Watt Benchmarking and Calculation}
%Introduction
For the experiment, it was decided to set up a series of benchmarks with devices with different CPU types, 
in which the computing power of devices is determined. 
While measuring the computing power by means of benchmarks, 
the energy consumption of the hardware is measured.
The two values obtained can then be used to calculate the ratio of the computing power to the energy consumed in the process. 

%Formula - Score per Watt
\begin{eqnarray}
    \frac{Score\: of\: Benchmark}{Power\: consumption\: [W]} = Score\: per\: Watt\: [W]
\end{eqnarray}

The resulting ratios of the two architectures can then be used to compare the processor types with each other and draw conclusions. 

With the experiment, measurable data is created that simulates a potential real-world use case. 
The resulting data of the experiment could later be used to draw conclusions and create a power-saving plan. 
With a power-saving plan, it can be decided which applications should run on which hardware in order to achieve the most power-efficient usage.
The creation of a dedicated power-saving plan and applying it is not the goal of this paper. 
It is limited to showing potential real-world benefits of choosing the best-suited CPU for certain tasks.


%Used Hardware
%Hardware Raspberry Pi 4
A Raspberry Pi 4 with 8 GB RAM is used as a representative for ARM CPUs.
It was released in May 2020 and is a single-board computer with 
limited expansion options.
The Random-Access Memory (RAM) can not be expanded and an SD card 
serves as the device's main storage.  
In addition, it offers four USB ports that can be used for expansions,
as well as an Ethernet port. 
There are also GPIO pins for connecting sensors and other devices.
It has a 64-bit 4-core Broadcom BCM2711 CPU with a maximum clock speed of 1,5GHz.
\cite{Raspberr44:online}

%Hardware Optiplex 7010
As the system contains an Intel-based CPU, an old Optiplex 7010 office PC is chosen.
This is a small form factor PC and was released in the year 2012. 
It has an Intel i7-3770 processor and 16 GB of RAM.
The Intel i7-3770 is a CPU released in 2012 with 4 cores and 8 threads
and has a clock speed of 3.4GHz which can be boosted to 3.9GHz.
\cite{IntelCor68:online}


\begin{table} [h!]
\centering
\caption{Comparison of Systems}
\begin{tabular}{| c | c | c |}
\hline
                    & Optiplex 7010             & Raspberry Pi 4 8GB                \\      \hline
Release             & 2012                      & 2020                              \\
CPU                 & Intel i7-3770             & Broadcom BCM2711                  \\    
Architecture        & x86                       & ARM 64                            \\ 
Cores               & 4                         & 4                                 \\    
Threads             & 8                         & 4                                 \\    
RAM                 & 16 GB                     & 8 GB                              \\    
Max Clock           & 3,9 GHz                   & 1,5 GHz                           \\      \hline
\end{tabular}
\end{table}

The significant difference between the hardware of both devices is not ideal.
The great differences between both systems introduce a lot of aspects, that can influence the outcome of the experiment.
For an optimal setup, both systems would share most of their components and
only differ from the used CPU and motherboard. 
This would greatly simplify the interpretation of the experiment and make it easier to draw conclusions about the CPU.
In the case of this experiment, it was not possible to gain access to such similar hardware. 
Nevertheless, this test setup can be used to obtain meaningful measured values, 
and conclusions can be drawn from the results. 


%Used Software - Operating System - Benchmarks
The operating system used to run both devices is Ubuntu Server 22.04 LTS without a desktop environment. 
This GNU/Linux distribution was chosen because it is one of the few that is available in the exact same version for both systems.
\cite{Canonica57:online}
The base configuration of the system is not altered, 
and only the benchmark software is installed additionally.
The two benchmark tools selected for the test are Sysbench and 7zip. 
These simulate real-world use cases, 
which can be used to evaluate the performance of the processors.
Each of the benchmarks is run 5 times on each device. Afterward, the average values of all measurements are created for creating a more stable end result of the tests.



%Benchmark 1 - 7-Zip
The first benchmark to be tested on the systems is the 7-Zip benchmark \cite{7zipBenchmarkLink}. 
This Benchmark tool evaluates CPU capabilities based on the
performance of compressing and decompressing data.
The benchmark uses input data which is stored 
on the main storage of the device.
The used data is then processed in the compression algorithm 
of 7-Zip and its output is stored back on the main storage.
Because of the involvement of other components in this test, 
those could also influence the outcome of the benchmark. 
For simplicity reasons only the values of the compression step were
taken into account and the average of the test results 
was used as the final result of each run.


%Benchmark 2 - Sysbench
The second benchmark tool is the CPU benchmark of 
sysbench \cite{sysbenchGithub}.
Sysbench provides many benchmark tests for various components of a system.
The CPU benchmark was chosen for this experiment. 
This benchmark program runs an algorithm to calculate
prime numbers within a certain time frame.
The result of the test is the achieved events per second of the system.
Which represents the score of the system.



%Used Tools
For the measurement of the consumed wattage during the tests,
a wattage and current meter were used. 
The device used for this was a Brennstuhl PM 231 E. 
This device is able to measure voltage, frequency,
current, power factor, and power of the device plugged into it.
\cite{PrimeraL19:online}
