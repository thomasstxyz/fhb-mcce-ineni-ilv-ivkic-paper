\begin{abstract}
The aim of this paper is to describe the idea of energy savings by using heterogeneous clusters and related ways to explore this.
There are different types of processors, which all have their advantages and disadvantages and in some scenarios are clearly superior to other models. This gave rise to the topic of this paper. By effectively exploiting the strengths of the processors used and thus increasing their effectiveness, it should be possible to save energy in the operation of software. This is to be implemented in a heterogeneous cluster, a cluster in which two different CPU architectures are operated.
For the research based on this paper two experiments will be performed. First, a heterogeneous cluster will be built to demonstrate the feasibility and the problems that arise during the process. Secondly, benchmarks will be performed on devices with the same processor architecture and the power consumption will be measured to calculate the performance per watt. This is necessary because the first test is to be carried out in the cloud and this data is not available. The resulting data should then provide further information about the usefulness of heterogeneous clusters. 
\end{abstract}
