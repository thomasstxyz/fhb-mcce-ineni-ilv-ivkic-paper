\section{Related Work}

\subsection{ARM architecture}

\citeauthor{maqbool2015evaluating} (\citeyear{maqbool2015evaluating})
state that in their benchmarks,
ARM performed four times better on single core workloads and
2.6 times better on four cores, 
regarding performance to power ratio.
On memory intensive database transactions however, 
x86 performed 12\% better for multithreaded query processing
\cite{maqbool2015evaluating}.

\citeauthor{ou2012energy} (\citeyear{ou2012energy}) analyzed ARM and Intel CPU cluster computing 
in regards to energy-efficiency. They measured different applications
like Web server, in-memory database and video transcoding. These
applications performed more energy-efficient in the ARM cluster than 
in the Intel workstation. Multiple ARM processors are needed to 
achieve performance like the Intel processor. 
The energy-efficiency ratio of the
ARM cluster against the Intel workstation varies from 2.6 - 9.5 in
in-memory database workload, to 1.3 in Web server workload, and 
1.21 in video transcoding. 
\citeauthor{ou2012energy} (\citeyear{ou2012energy}) also state that the Intel processor
uses a dynamic voltage and frequency scaling technique, which 
causes the power consumption to not be linear to the CPU utilization level.
For the ARM processor, this is not the case.
Furthermore \citeauthor{ou2012energy} (\citeyear{ou2012energy}) concluded that ARM cluster 
based data centers are advantageous from a cost perspective
in computationally lightweight applications,
however when it comes to computation-intensive applications
like dynamic Web server application or video transcoding, the advantages
of ARM processors decline gradually, as the number of ARM processors needed 
to provide comparable performance ramps up
\cite{ou2012energy}.