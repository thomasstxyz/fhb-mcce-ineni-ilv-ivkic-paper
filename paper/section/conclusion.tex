\section{Conclusion}
% Conclusion
This paper gives an overview of possible application areas and advantages of cleverly used heterogeneous clusters.  
However, many of the points raised still need to be discussed further, 
and the current possibilities of implementing a heterogeneous cluster need to be evaluated in a real-world scenario.

The results have shown significant differences in efficiency in the tested use cases in the experiment.
Where in the 7-Zip benchmark the difference in efficiency was by a factor of 2.2, the difference in the Sysbench benchmark was more than 11 times higher.
For the interpretation of the results, this means that certain types of software can be run significantly more efficiently if the right hardware is chosen. 
This means having the ability to choose which system an application runs can decrease the overall power consumption.

An important thing to note about the experiment is the less than optimal hardware that was used for the benchmarks.
The comparison of a single-board computer with a full desktop PC is not completely representative of a comparison between two processor architectures. Also, the additional hardware in a desktop PC may introduce influences and disturbing
factors during the experiment.

To further investigate the operation of
heterogeneous clusters in more detail, 
a prototype is presented in the future work section of this paper.
The structure and goal of the investigations were described
in this paper and can be implemented and documented 
in a subsequent paper so that conclusions can be drawn from
the results about the usefulness of using heterogeneous
clusters with today's technology.
